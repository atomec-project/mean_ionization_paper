\documentclass[11pt]{article}

    \usepackage[breakable]{tcolorbox}
    \usepackage{parskip} % Stop auto-indenting (to mimic markdown behaviour)
    
    \usepackage{iftex}
    \ifPDFTeX
    	\usepackage[T1]{fontenc}
    	\usepackage{mathpazo}
    \else
    	\usepackage{fontspec}
    \fi

    % Basic figure setup, for now with no caption control since it's done
    % automatically by Pandoc (which extracts ![](path) syntax from Markdown).
    \usepackage{graphicx}
    % Maintain compatibility with old templates. Remove in nbconvert 6.0
    \let\Oldincludegraphics\includegraphics
    % Ensure that by default, figures have no caption (until we provide a
    % proper Figure object with a Caption API and a way to capture that
    % in the conversion process - todo).
    \usepackage{caption}
    \DeclareCaptionFormat{nocaption}{}
    \captionsetup{format=nocaption,aboveskip=0pt,belowskip=0pt}

    \usepackage{float}
    \floatplacement{figure}{H} % forces figures to be placed at the correct location
    \usepackage{xcolor} % Allow colors to be defined
    \usepackage{enumerate} % Needed for markdown enumerations to work
    \usepackage{geometry} % Used to adjust the document margins
    \usepackage{amsmath} % Equations
    \usepackage{amssymb} % Equations
    \usepackage{textcomp} % defines textquotesingle
    % Hack from http://tex.stackexchange.com/a/47451/13684:
    \AtBeginDocument{%
        \def\PYZsq{\textquotesingle}% Upright quotes in Pygmentized code
    }
    \usepackage{upquote} % Upright quotes for verbatim code
    \usepackage{eurosym} % defines \euro
    \usepackage[mathletters]{ucs} % Extended unicode (utf-8) support
    \usepackage{fancyvrb} % verbatim replacement that allows latex
    \usepackage{grffile} % extends the file name processing of package graphics 
                         % to support a larger range
    \makeatletter % fix for old versions of grffile with XeLaTeX
    \@ifpackagelater{grffile}{2019/11/01}
    {
      % Do nothing on new versions
    }
    {
      \def\Gread@@xetex#1{%
        \IfFileExists{"\Gin@base".bb}%
        {\Gread@eps{\Gin@base.bb}}%
        {\Gread@@xetex@aux#1}%
      }
    }
    \makeatother
    \usepackage[Export]{adjustbox} % Used to constrain images to a maximum size
    \adjustboxset{max size={0.9\linewidth}{0.9\paperheight}}

    % The hyperref package gives us a pdf with properly built
    % internal navigation ('pdf bookmarks' for the table of contents,
    % internal cross-reference links, web links for URLs, etc.)
    \usepackage{hyperref}
    % The default LaTeX title has an obnoxious amount of whitespace. By default,
    % titling removes some of it. It also provides customization options.
    \usepackage{titling}
    \usepackage{longtable} % longtable support required by pandoc >1.10
    \usepackage{booktabs}  % table support for pandoc > 1.12.2
    \usepackage[inline]{enumitem} % IRkernel/repr support (it uses the enumerate* environment)
    \usepackage[normalem]{ulem} % ulem is needed to support strikethroughs (\sout)
                                % normalem makes italics be italics, not underlines
    \usepackage{mathrsfs}
    

    
    % Colors for the hyperref package
    \definecolor{urlcolor}{rgb}{0,.145,.698}
    \definecolor{linkcolor}{rgb}{.71,0.21,0.01}
    \definecolor{citecolor}{rgb}{.12,.54,.11}

    % ANSI colors
    \definecolor{ansi-black}{HTML}{3E424D}
    \definecolor{ansi-black-intense}{HTML}{282C36}
    \definecolor{ansi-red}{HTML}{E75C58}
    \definecolor{ansi-red-intense}{HTML}{B22B31}
    \definecolor{ansi-green}{HTML}{00A250}
    \definecolor{ansi-green-intense}{HTML}{007427}
    \definecolor{ansi-yellow}{HTML}{DDB62B}
    \definecolor{ansi-yellow-intense}{HTML}{B27D12}
    \definecolor{ansi-blue}{HTML}{208FFB}
    \definecolor{ansi-blue-intense}{HTML}{0065CA}
    \definecolor{ansi-magenta}{HTML}{D160C4}
    \definecolor{ansi-magenta-intense}{HTML}{A03196}
    \definecolor{ansi-cyan}{HTML}{60C6C8}
    \definecolor{ansi-cyan-intense}{HTML}{258F8F}
    \definecolor{ansi-white}{HTML}{C5C1B4}
    \definecolor{ansi-white-intense}{HTML}{A1A6B2}
    \definecolor{ansi-default-inverse-fg}{HTML}{FFFFFF}
    \definecolor{ansi-default-inverse-bg}{HTML}{000000}

    % common color for the border for error outputs.
    \definecolor{outerrorbackground}{HTML}{FFDFDF}

    % commands and environments needed by pandoc snippets
    % extracted from the output of `pandoc -s`
    \providecommand{\tightlist}{%
      \setlength{\itemsep}{0pt}\setlength{\parskip}{0pt}}
    \DefineVerbatimEnvironment{Highlighting}{Verbatim}{commandchars=\\\{\}}
    % Add ',fontsize=\small' for more characters per line
    \newenvironment{Shaded}{}{}
    \newcommand{\KeywordTok}[1]{\textcolor[rgb]{0.00,0.44,0.13}{\textbf{{#1}}}}
    \newcommand{\DataTypeTok}[1]{\textcolor[rgb]{0.56,0.13,0.00}{{#1}}}
    \newcommand{\DecValTok}[1]{\textcolor[rgb]{0.25,0.63,0.44}{{#1}}}
    \newcommand{\BaseNTok}[1]{\textcolor[rgb]{0.25,0.63,0.44}{{#1}}}
    \newcommand{\FloatTok}[1]{\textcolor[rgb]{0.25,0.63,0.44}{{#1}}}
    \newcommand{\CharTok}[1]{\textcolor[rgb]{0.25,0.44,0.63}{{#1}}}
    \newcommand{\StringTok}[1]{\textcolor[rgb]{0.25,0.44,0.63}{{#1}}}
    \newcommand{\CommentTok}[1]{\textcolor[rgb]{0.38,0.63,0.69}{\textit{{#1}}}}
    \newcommand{\OtherTok}[1]{\textcolor[rgb]{0.00,0.44,0.13}{{#1}}}
    \newcommand{\AlertTok}[1]{\textcolor[rgb]{1.00,0.00,0.00}{\textbf{{#1}}}}
    \newcommand{\FunctionTok}[1]{\textcolor[rgb]{0.02,0.16,0.49}{{#1}}}
    \newcommand{\RegionMarkerTok}[1]{{#1}}
    \newcommand{\ErrorTok}[1]{\textcolor[rgb]{1.00,0.00,0.00}{\textbf{{#1}}}}
    \newcommand{\NormalTok}[1]{{#1}}
    
    % Additional commands for more recent versions of Pandoc
    \newcommand{\ConstantTok}[1]{\textcolor[rgb]{0.53,0.00,0.00}{{#1}}}
    \newcommand{\SpecialCharTok}[1]{\textcolor[rgb]{0.25,0.44,0.63}{{#1}}}
    \newcommand{\VerbatimStringTok}[1]{\textcolor[rgb]{0.25,0.44,0.63}{{#1}}}
    \newcommand{\SpecialStringTok}[1]{\textcolor[rgb]{0.73,0.40,0.53}{{#1}}}
    \newcommand{\ImportTok}[1]{{#1}}
    \newcommand{\DocumentationTok}[1]{\textcolor[rgb]{0.73,0.13,0.13}{\textit{{#1}}}}
    \newcommand{\AnnotationTok}[1]{\textcolor[rgb]{0.38,0.63,0.69}{\textbf{\textit{{#1}}}}}
    \newcommand{\CommentVarTok}[1]{\textcolor[rgb]{0.38,0.63,0.69}{\textbf{\textit{{#1}}}}}
    \newcommand{\VariableTok}[1]{\textcolor[rgb]{0.10,0.09,0.49}{{#1}}}
    \newcommand{\ControlFlowTok}[1]{\textcolor[rgb]{0.00,0.44,0.13}{\textbf{{#1}}}}
    \newcommand{\OperatorTok}[1]{\textcolor[rgb]{0.40,0.40,0.40}{{#1}}}
    \newcommand{\BuiltInTok}[1]{{#1}}
    \newcommand{\ExtensionTok}[1]{{#1}}
    \newcommand{\PreprocessorTok}[1]{\textcolor[rgb]{0.74,0.48,0.00}{{#1}}}
    \newcommand{\AttributeTok}[1]{\textcolor[rgb]{0.49,0.56,0.16}{{#1}}}
    \newcommand{\InformationTok}[1]{\textcolor[rgb]{0.38,0.63,0.69}{\textbf{\textit{{#1}}}}}
    \newcommand{\WarningTok}[1]{\textcolor[rgb]{0.38,0.63,0.69}{\textbf{\textit{{#1}}}}}
    
    
    % Define a nice break command that doesn't care if a line doesn't already
    % exist.
    \def\br{\hspace*{\fill} \\* }
    % Math Jax compatibility definitions
    \def\gt{>}
    \def\lt{<}
    \let\Oldtex\TeX
    \let\Oldlatex\LaTeX
    \renewcommand{\TeX}{\textrm{\Oldtex}}
    \renewcommand{\LaTeX}{\textrm{\Oldlatex}}
    % Document parameters
    % Document title
    \title{Supplementary information}
    
    
    
    
    
% Pygments definitions
\makeatletter
\def\PY@reset{\let\PY@it=\relax \let\PY@bf=\relax%
    \let\PY@ul=\relax \let\PY@tc=\relax%
    \let\PY@bc=\relax \let\PY@ff=\relax}
\def\PY@tok#1{\csname PY@tok@#1\endcsname}
\def\PY@toks#1+{\ifx\relax#1\empty\else%
    \PY@tok{#1}\expandafter\PY@toks\fi}
\def\PY@do#1{\PY@bc{\PY@tc{\PY@ul{%
    \PY@it{\PY@bf{\PY@ff{#1}}}}}}}
\def\PY#1#2{\PY@reset\PY@toks#1+\relax+\PY@do{#2}}

\@namedef{PY@tok@w}{\def\PY@tc##1{\textcolor[rgb]{0.73,0.73,0.73}{##1}}}
\@namedef{PY@tok@c}{\let\PY@it=\textit\def\PY@tc##1{\textcolor[rgb]{0.25,0.50,0.50}{##1}}}
\@namedef{PY@tok@cp}{\def\PY@tc##1{\textcolor[rgb]{0.74,0.48,0.00}{##1}}}
\@namedef{PY@tok@k}{\let\PY@bf=\textbf\def\PY@tc##1{\textcolor[rgb]{0.00,0.50,0.00}{##1}}}
\@namedef{PY@tok@kp}{\def\PY@tc##1{\textcolor[rgb]{0.00,0.50,0.00}{##1}}}
\@namedef{PY@tok@kt}{\def\PY@tc##1{\textcolor[rgb]{0.69,0.00,0.25}{##1}}}
\@namedef{PY@tok@o}{\def\PY@tc##1{\textcolor[rgb]{0.40,0.40,0.40}{##1}}}
\@namedef{PY@tok@ow}{\let\PY@bf=\textbf\def\PY@tc##1{\textcolor[rgb]{0.67,0.13,1.00}{##1}}}
\@namedef{PY@tok@nb}{\def\PY@tc##1{\textcolor[rgb]{0.00,0.50,0.00}{##1}}}
\@namedef{PY@tok@nf}{\def\PY@tc##1{\textcolor[rgb]{0.00,0.00,1.00}{##1}}}
\@namedef{PY@tok@nc}{\let\PY@bf=\textbf\def\PY@tc##1{\textcolor[rgb]{0.00,0.00,1.00}{##1}}}
\@namedef{PY@tok@nn}{\let\PY@bf=\textbf\def\PY@tc##1{\textcolor[rgb]{0.00,0.00,1.00}{##1}}}
\@namedef{PY@tok@ne}{\let\PY@bf=\textbf\def\PY@tc##1{\textcolor[rgb]{0.82,0.25,0.23}{##1}}}
\@namedef{PY@tok@nv}{\def\PY@tc##1{\textcolor[rgb]{0.10,0.09,0.49}{##1}}}
\@namedef{PY@tok@no}{\def\PY@tc##1{\textcolor[rgb]{0.53,0.00,0.00}{##1}}}
\@namedef{PY@tok@nl}{\def\PY@tc##1{\textcolor[rgb]{0.63,0.63,0.00}{##1}}}
\@namedef{PY@tok@ni}{\let\PY@bf=\textbf\def\PY@tc##1{\textcolor[rgb]{0.60,0.60,0.60}{##1}}}
\@namedef{PY@tok@na}{\def\PY@tc##1{\textcolor[rgb]{0.49,0.56,0.16}{##1}}}
\@namedef{PY@tok@nt}{\let\PY@bf=\textbf\def\PY@tc##1{\textcolor[rgb]{0.00,0.50,0.00}{##1}}}
\@namedef{PY@tok@nd}{\def\PY@tc##1{\textcolor[rgb]{0.67,0.13,1.00}{##1}}}
\@namedef{PY@tok@s}{\def\PY@tc##1{\textcolor[rgb]{0.73,0.13,0.13}{##1}}}
\@namedef{PY@tok@sd}{\let\PY@it=\textit\def\PY@tc##1{\textcolor[rgb]{0.73,0.13,0.13}{##1}}}
\@namedef{PY@tok@si}{\let\PY@bf=\textbf\def\PY@tc##1{\textcolor[rgb]{0.73,0.40,0.53}{##1}}}
\@namedef{PY@tok@se}{\let\PY@bf=\textbf\def\PY@tc##1{\textcolor[rgb]{0.73,0.40,0.13}{##1}}}
\@namedef{PY@tok@sr}{\def\PY@tc##1{\textcolor[rgb]{0.73,0.40,0.53}{##1}}}
\@namedef{PY@tok@ss}{\def\PY@tc##1{\textcolor[rgb]{0.10,0.09,0.49}{##1}}}
\@namedef{PY@tok@sx}{\def\PY@tc##1{\textcolor[rgb]{0.00,0.50,0.00}{##1}}}
\@namedef{PY@tok@m}{\def\PY@tc##1{\textcolor[rgb]{0.40,0.40,0.40}{##1}}}
\@namedef{PY@tok@gh}{\let\PY@bf=\textbf\def\PY@tc##1{\textcolor[rgb]{0.00,0.00,0.50}{##1}}}
\@namedef{PY@tok@gu}{\let\PY@bf=\textbf\def\PY@tc##1{\textcolor[rgb]{0.50,0.00,0.50}{##1}}}
\@namedef{PY@tok@gd}{\def\PY@tc##1{\textcolor[rgb]{0.63,0.00,0.00}{##1}}}
\@namedef{PY@tok@gi}{\def\PY@tc##1{\textcolor[rgb]{0.00,0.63,0.00}{##1}}}
\@namedef{PY@tok@gr}{\def\PY@tc##1{\textcolor[rgb]{1.00,0.00,0.00}{##1}}}
\@namedef{PY@tok@ge}{\let\PY@it=\textit}
\@namedef{PY@tok@gs}{\let\PY@bf=\textbf}
\@namedef{PY@tok@gp}{\let\PY@bf=\textbf\def\PY@tc##1{\textcolor[rgb]{0.00,0.00,0.50}{##1}}}
\@namedef{PY@tok@go}{\def\PY@tc##1{\textcolor[rgb]{0.53,0.53,0.53}{##1}}}
\@namedef{PY@tok@gt}{\def\PY@tc##1{\textcolor[rgb]{0.00,0.27,0.87}{##1}}}
\@namedef{PY@tok@err}{\def\PY@bc##1{{\setlength{\fboxsep}{\string -\fboxrule}\fcolorbox[rgb]{1.00,0.00,0.00}{1,1,1}{\strut ##1}}}}
\@namedef{PY@tok@kc}{\let\PY@bf=\textbf\def\PY@tc##1{\textcolor[rgb]{0.00,0.50,0.00}{##1}}}
\@namedef{PY@tok@kd}{\let\PY@bf=\textbf\def\PY@tc##1{\textcolor[rgb]{0.00,0.50,0.00}{##1}}}
\@namedef{PY@tok@kn}{\let\PY@bf=\textbf\def\PY@tc##1{\textcolor[rgb]{0.00,0.50,0.00}{##1}}}
\@namedef{PY@tok@kr}{\let\PY@bf=\textbf\def\PY@tc##1{\textcolor[rgb]{0.00,0.50,0.00}{##1}}}
\@namedef{PY@tok@bp}{\def\PY@tc##1{\textcolor[rgb]{0.00,0.50,0.00}{##1}}}
\@namedef{PY@tok@fm}{\def\PY@tc##1{\textcolor[rgb]{0.00,0.00,1.00}{##1}}}
\@namedef{PY@tok@vc}{\def\PY@tc##1{\textcolor[rgb]{0.10,0.09,0.49}{##1}}}
\@namedef{PY@tok@vg}{\def\PY@tc##1{\textcolor[rgb]{0.10,0.09,0.49}{##1}}}
\@namedef{PY@tok@vi}{\def\PY@tc##1{\textcolor[rgb]{0.10,0.09,0.49}{##1}}}
\@namedef{PY@tok@vm}{\def\PY@tc##1{\textcolor[rgb]{0.10,0.09,0.49}{##1}}}
\@namedef{PY@tok@sa}{\def\PY@tc##1{\textcolor[rgb]{0.73,0.13,0.13}{##1}}}
\@namedef{PY@tok@sb}{\def\PY@tc##1{\textcolor[rgb]{0.73,0.13,0.13}{##1}}}
\@namedef{PY@tok@sc}{\def\PY@tc##1{\textcolor[rgb]{0.73,0.13,0.13}{##1}}}
\@namedef{PY@tok@dl}{\def\PY@tc##1{\textcolor[rgb]{0.73,0.13,0.13}{##1}}}
\@namedef{PY@tok@s2}{\def\PY@tc##1{\textcolor[rgb]{0.73,0.13,0.13}{##1}}}
\@namedef{PY@tok@sh}{\def\PY@tc##1{\textcolor[rgb]{0.73,0.13,0.13}{##1}}}
\@namedef{PY@tok@s1}{\def\PY@tc##1{\textcolor[rgb]{0.73,0.13,0.13}{##1}}}
\@namedef{PY@tok@mb}{\def\PY@tc##1{\textcolor[rgb]{0.40,0.40,0.40}{##1}}}
\@namedef{PY@tok@mf}{\def\PY@tc##1{\textcolor[rgb]{0.40,0.40,0.40}{##1}}}
\@namedef{PY@tok@mh}{\def\PY@tc##1{\textcolor[rgb]{0.40,0.40,0.40}{##1}}}
\@namedef{PY@tok@mi}{\def\PY@tc##1{\textcolor[rgb]{0.40,0.40,0.40}{##1}}}
\@namedef{PY@tok@il}{\def\PY@tc##1{\textcolor[rgb]{0.40,0.40,0.40}{##1}}}
\@namedef{PY@tok@mo}{\def\PY@tc##1{\textcolor[rgb]{0.40,0.40,0.40}{##1}}}
\@namedef{PY@tok@ch}{\let\PY@it=\textit\def\PY@tc##1{\textcolor[rgb]{0.25,0.50,0.50}{##1}}}
\@namedef{PY@tok@cm}{\let\PY@it=\textit\def\PY@tc##1{\textcolor[rgb]{0.25,0.50,0.50}{##1}}}
\@namedef{PY@tok@cpf}{\let\PY@it=\textit\def\PY@tc##1{\textcolor[rgb]{0.25,0.50,0.50}{##1}}}
\@namedef{PY@tok@c1}{\let\PY@it=\textit\def\PY@tc##1{\textcolor[rgb]{0.25,0.50,0.50}{##1}}}
\@namedef{PY@tok@cs}{\let\PY@it=\textit\def\PY@tc##1{\textcolor[rgb]{0.25,0.50,0.50}{##1}}}

\def\PYZbs{\char`\\}
\def\PYZus{\char`\_}
\def\PYZob{\char`\{}
\def\PYZcb{\char`\}}
\def\PYZca{\char`\^}
\def\PYZam{\char`\&}
\def\PYZlt{\char`\<}
\def\PYZgt{\char`\>}
\def\PYZsh{\char`\#}
\def\PYZpc{\char`\%}
\def\PYZdl{\char`\$}
\def\PYZhy{\char`\-}
\def\PYZsq{\char`\'}
\def\PYZdq{\char`\"}
\def\PYZti{\char`\~}
% for compatibility with earlier versions
\def\PYZat{@}
\def\PYZlb{[}
\def\PYZrb{]}
\makeatother


    % For linebreaks inside Verbatim environment from package fancyvrb. 
    \makeatletter
        \newbox\Wrappedcontinuationbox 
        \newbox\Wrappedvisiblespacebox 
        \newcommand*\Wrappedvisiblespace {\textcolor{red}{\textvisiblespace}} 
        \newcommand*\Wrappedcontinuationsymbol {\textcolor{red}{\llap{\tiny$\m@th\hookrightarrow$}}} 
        \newcommand*\Wrappedcontinuationindent {3ex } 
        \newcommand*\Wrappedafterbreak {\kern\Wrappedcontinuationindent\copy\Wrappedcontinuationbox} 
        % Take advantage of the already applied Pygments mark-up to insert 
        % potential linebreaks for TeX processing. 
        %        {, <, #, %, $, ' and ": go to next line. 
        %        _, }, ^, &, >, - and ~: stay at end of broken line. 
        % Use of \textquotesingle for straight quote. 
        \newcommand*\Wrappedbreaksatspecials {% 
            \def\PYGZus{\discretionary{\char`\_}{\Wrappedafterbreak}{\char`\_}}% 
            \def\PYGZob{\discretionary{}{\Wrappedafterbreak\char`\{}{\char`\{}}% 
            \def\PYGZcb{\discretionary{\char`\}}{\Wrappedafterbreak}{\char`\}}}% 
            \def\PYGZca{\discretionary{\char`\^}{\Wrappedafterbreak}{\char`\^}}% 
            \def\PYGZam{\discretionary{\char`\&}{\Wrappedafterbreak}{\char`\&}}% 
            \def\PYGZlt{\discretionary{}{\Wrappedafterbreak\char`\<}{\char`\<}}% 
            \def\PYGZgt{\discretionary{\char`\>}{\Wrappedafterbreak}{\char`\>}}% 
            \def\PYGZsh{\discretionary{}{\Wrappedafterbreak\char`\#}{\char`\#}}% 
            \def\PYGZpc{\discretionary{}{\Wrappedafterbreak\char`\%}{\char`\%}}% 
            \def\PYGZdl{\discretionary{}{\Wrappedafterbreak\char`\$}{\char`\$}}% 
            \def\PYGZhy{\discretionary{\char`\-}{\Wrappedafterbreak}{\char`\-}}% 
            \def\PYGZsq{\discretionary{}{\Wrappedafterbreak\textquotesingle}{\textquotesingle}}% 
            \def\PYGZdq{\discretionary{}{\Wrappedafterbreak\char`\"}{\char`\"}}% 
            \def\PYGZti{\discretionary{\char`\~}{\Wrappedafterbreak}{\char`\~}}% 
        } 
        % Some characters . , ; ? ! / are not pygmentized. 
        % This macro makes them "active" and they will insert potential linebreaks 
        \newcommand*\Wrappedbreaksatpunct {% 
            \lccode`\~`\.\lowercase{\def~}{\discretionary{\hbox{\char`\.}}{\Wrappedafterbreak}{\hbox{\char`\.}}}% 
            \lccode`\~`\,\lowercase{\def~}{\discretionary{\hbox{\char`\,}}{\Wrappedafterbreak}{\hbox{\char`\,}}}% 
            \lccode`\~`\;\lowercase{\def~}{\discretionary{\hbox{\char`\;}}{\Wrappedafterbreak}{\hbox{\char`\;}}}% 
            \lccode`\~`\:\lowercase{\def~}{\discretionary{\hbox{\char`\:}}{\Wrappedafterbreak}{\hbox{\char`\:}}}% 
            \lccode`\~`\?\lowercase{\def~}{\discretionary{\hbox{\char`\?}}{\Wrappedafterbreak}{\hbox{\char`\?}}}% 
            \lccode`\~`\!\lowercase{\def~}{\discretionary{\hbox{\char`\!}}{\Wrappedafterbreak}{\hbox{\char`\!}}}% 
            \lccode`\~`\/\lowercase{\def~}{\discretionary{\hbox{\char`\/}}{\Wrappedafterbreak}{\hbox{\char`\/}}}% 
            \catcode`\.\active
            \catcode`\,\active 
            \catcode`\;\active
            \catcode`\:\active
            \catcode`\?\active
            \catcode`\!\active
            \catcode`\/\active 
            \lccode`\~`\~ 	
        }
    \makeatother

    \let\OriginalVerbatim=\Verbatim
    \makeatletter
    \renewcommand{\Verbatim}[1][1]{%
        %\parskip\z@skip
        \sbox\Wrappedcontinuationbox {\Wrappedcontinuationsymbol}%
        \sbox\Wrappedvisiblespacebox {\FV@SetupFont\Wrappedvisiblespace}%
        \def\FancyVerbFormatLine ##1{\hsize\linewidth
            \vtop{\raggedright\hyphenpenalty\z@\exhyphenpenalty\z@
                \doublehyphendemerits\z@\finalhyphendemerits\z@
                \strut ##1\strut}%
        }%
        % If the linebreak is at a space, the latter will be displayed as visible
        % space at end of first line, and a continuation symbol starts next line.
        % Stretch/shrink are however usually zero for typewriter font.
        \def\FV@Space {%
            \nobreak\hskip\z@ plus\fontdimen3\font minus\fontdimen4\font
            \discretionary{\copy\Wrappedvisiblespacebox}{\Wrappedafterbreak}
            {\kern\fontdimen2\font}%
        }%
        
        % Allow breaks at special characters using \PYG... macros.
        \Wrappedbreaksatspecials
        % Breaks at punctuation characters . , ; ? ! and / need catcode=\active 	
        \OriginalVerbatim[#1,codes*=\Wrappedbreaksatpunct]%
    }
    \makeatother

    % Exact colors from NB
    \definecolor{incolor}{HTML}{303F9F}
    \definecolor{outcolor}{HTML}{D84315}
    \definecolor{cellborder}{HTML}{CFCFCF}
    \definecolor{cellbackground}{HTML}{F7F7F7}
    
    % prompt
    \makeatletter
    \newcommand{\boxspacing}{\kern\kvtcb@left@rule\kern\kvtcb@boxsep}
    \makeatother
    \newcommand{\prompt}[4]{
        {\ttfamily\llap{{\color{#2}[#3]:\hspace{3pt}#4}}\vspace{-\baselineskip}}
    }
    

    
    % Prevent overflowing lines due to hard-to-break entities
    \sloppy 
    % Setup hyperref package
    \hypersetup{
      breaklinks=true,  % so long urls are correctly broken across lines
      colorlinks=true,
      urlcolor=urlcolor,
      linkcolor=linkcolor,
      citecolor=citecolor,
      }
    % Slightly bigger margins than the latex defaults
    
    \geometry{verbose,tmargin=1in,bmargin=1in,lmargin=1in,rmargin=1in}
    
    

\begin{document}
    
    \maketitle
    
    

    
    \hypertarget{introduction-to-atomec}{%
\section{Introduction to atoMEC}\label{introduction-to-atomec}}

For all calculations in the paper, we have used the open source
average-atom code atoMEC. It can be downloaded
\href{https://github.com/atomec-project/atoMEC}{here}.

In this section, we go through the basics of setting up and then running
a calcuation in atoMEC.

We first create an \texttt{Atom} object which houses the key physical
information about the system we want to study, i.e.~the temperature and
mass density. We start with Aluminium at room temperature:

    \begin{tcolorbox}[breakable, size=fbox, boxrule=1pt, pad at break*=1mm,colback=cellbackground, colframe=cellborder]
\prompt{In}{incolor}{1}{\boxspacing}
\begin{Verbatim}[commandchars=\\\{\}]
\PY{c+c1}{\PYZsh{} import the Atom object}
\PY{k+kn}{from} \PY{n+nn}{atoMEC} \PY{k+kn}{import} \PY{n}{Atom}

\PY{c+c1}{\PYZsh{} set up the Aluminium atom}
\PY{n}{Al\PYZus{}atom} \PY{o}{=} \PY{n}{Atom}\PY{p}{(}\PY{l+s+s2}{\PYZdq{}}\PY{l+s+s2}{Al}\PY{l+s+s2}{\PYZdq{}}\PY{p}{,} \PY{l+m+mi}{300}\PY{p}{,} \PY{n}{density}\PY{o}{=}\PY{l+m+mf}{2.7}\PY{p}{,} \PY{n}{units\PYZus{}temp}\PY{o}{=}\PY{l+s+s2}{\PYZdq{}}\PY{l+s+s2}{K}\PY{l+s+s2}{\PYZdq{}}\PY{p}{)}
\end{Verbatim}
\end{tcolorbox}

    \begin{Verbatim}[commandchars=\\\{\}]

Welcome to atoMEC!

Warning: this input temperature is very low. Proceeding anyway, but results may
not be accurate.
Normal temperature range for atoMEC is 0.01 -- 100 eV

Atomic information:

Atomic species                : Al
Atomic charge / weight        : 13  / 26.982
Valence electrons             : 3
Mass density                  : 2.7 g cm\^{}-3
Voronoi sphere radius         : 2.997 Bohr / 1.586 Angstrom
Electronic temperature        : 0.00095 Ha /  0.02585 eV / 300 K
Wigner-Seitz radius           : 2.078 (Bohr)
Ionic coupling parameter      : 2.967e+04
Electron degeneracy parameter : 0.002228


    \end{Verbatim}

    We see this prints some key information about our system, including for
example the ionic coupling and electron degeneracy parameters, which are
important in warm dense matter (WDM). For details of how these are
computed see our initial
\href{https://arxiv.org/abs/2103.09928}{preprint} on AA models and
documentation in the code.

Next we set up a \texttt{model} object, which contains input regarding
which approximations we use in our model, for example the boundary
condition and exchange-correlation (XC) approximation.

    \begin{tcolorbox}[breakable, size=fbox, boxrule=1pt, pad at break*=1mm,colback=cellbackground, colframe=cellborder]
\prompt{In}{incolor}{2}{\boxspacing}
\begin{Verbatim}[commandchars=\\\{\}]
\PY{c+c1}{\PYZsh{} import models}
\PY{k+kn}{from} \PY{n+nn}{atoMEC} \PY{k+kn}{import} \PY{n}{models}

\PY{c+c1}{\PYZsh{} set up the ISModel}
\PY{n}{Al\PYZus{}model} \PY{o}{=} \PY{n}{models}\PY{o}{.}\PY{n}{ISModel}\PY{p}{(}
    \PY{n}{Al\PYZus{}atom}\PY{p}{,} \PY{n}{bc}\PY{o}{=}\PY{l+s+s2}{\PYZdq{}}\PY{l+s+s2}{dirichlet}\PY{l+s+s2}{\PYZdq{}}\PY{p}{,} \PY{n}{unbound}\PY{o}{=}\PY{l+s+s2}{\PYZdq{}}\PY{l+s+s2}{quantum}\PY{l+s+s2}{\PYZdq{}}\PY{p}{,} \PY{n}{xfunc\PYZus{}id}\PY{o}{=}\PY{l+s+s2}{\PYZdq{}}\PY{l+s+s2}{lda\PYZus{}x}\PY{l+s+s2}{\PYZdq{}}\PY{p}{,} \PY{n}{cfunc\PYZus{}id}\PY{o}{=}\PY{l+s+s2}{\PYZdq{}}\PY{l+s+s2}{lda\PYZus{}c\PYZus{}pw}\PY{l+s+s2}{\PYZdq{}}
\PY{p}{)}
\end{Verbatim}
\end{tcolorbox}

    \begin{Verbatim}[commandchars=\\\{\}]
Using Ion-Sphere model
Ion-sphere model parameters:

Spin-polarized                : False
Number of electrons           : 13
Exchange functional           : lda\_x
Correlation functional        : lda\_c\_pw
Boundary condition            : dirichlet
Unbound electron treatment    : quantum
Shift KS potential            : True


    \end{Verbatim}

    In the above, we have set up an ion-sphere model (\texttt{ISModel})
which so far is the only kind of AA model implemented in atoMEC.
Furthermore, we have specified the following approximations:

\begin{itemize}
\tightlist
\item
  \texttt{unbound="quantum"}: This means all our KS orbitals are treated
  in the same way, regardless of their energy
\item
  \texttt{bc="dirichlet"}: The Dirichlet boundary condition (as
  described in the main text) is applied to the orbitals
\item
  \texttt{xfunc\_id="lda\_x"}, \texttt{cfunc\_id="lda\_c\_pw"}: We have
  chosen the LDA XC functional
\end{itemize}

We are now ready to run an SCF calculation, which is done by the
\texttt{CalcEnergy} function. There are various inputs to this function
which control numerical aspects, such as the number of grid points and
SCF convergence parameters. Most of these are optional so we use the
default values for now.

The two parameters which must be specified are the maximal value of the
principal and angular quantum numbers, \texttt{nmax} and \texttt{lmax}.
atoMEC will search for all the eigenvalues in the range
\(0<n<\textrm{nmax}\), \(0<l<\textrm{lmax}\). Since we have a system at
room temperature we do not need to include lots of states so we set
\texttt{nmax=5}, \texttt{lmax=3}.

    \begin{tcolorbox}[breakable, size=fbox, boxrule=1pt, pad at break*=1mm,colback=cellbackground, colframe=cellborder]
\prompt{In}{incolor}{3}{\boxspacing}
\begin{Verbatim}[commandchars=\\\{\}]
\PY{c+c1}{\PYZsh{} set the values of nmax and lmax}
\PY{n}{nmax} \PY{o}{=} \PY{l+m+mi}{5}
\PY{n}{lmax} \PY{o}{=} \PY{l+m+mi}{3}

\PY{c+c1}{\PYZsh{} run the SCF calculation}
\PY{n}{output} \PY{o}{=} \PY{n}{Al\PYZus{}model}\PY{o}{.}\PY{n}{CalcEnergy}\PY{p}{(}\PY{n}{nmax}\PY{p}{,} \PY{n}{lmax}\PY{p}{,} \PY{n}{scf\PYZus{}params}\PY{o}{=}\PY{p}{\PYZob{}}\PY{l+s+s2}{\PYZdq{}}\PY{l+s+s2}{mixfrac}\PY{l+s+s2}{\PYZdq{}}\PY{p}{:} \PY{l+m+mf}{0.6}\PY{p}{,} \PY{l+s+s2}{\PYZdq{}}\PY{l+s+s2}{maxscf}\PY{l+s+s2}{\PYZdq{}}\PY{p}{:} \PY{l+m+mi}{50}\PY{p}{\PYZcb{}}\PY{p}{)}
\end{Verbatim}
\end{tcolorbox}

    \begin{Verbatim}[commandchars=\\\{\}]
Starting SCF energy calculation

iscf   E\_free (Ha)    dE (1.0e-05)   dn (1.0e-04)   dv (1.0e-04)
-----------------------------------------------------------------
   0   -210.9503289      1.000e+00      9.999e-01      1.000e+00
   1   -222.7062710      5.279e-02      1.006e+00      7.357e-01
   2   -235.5939281      5.470e-02      7.726e-01      3.973e-01
   3   -240.4608125      2.024e-02      1.988e-01      6.811e-02
   4   -240.5743831      4.721e-04      3.588e-02      1.554e-02
   5   -240.5758261      5.998e-06      3.137e-03      6.085e-03
   6   -240.5760477      9.209e-07      1.177e-03      2.093e-03
   7   -240.5759785      2.877e-07      4.315e-04      1.424e-03
   8   -240.5759149      2.642e-07      1.589e-04      4.043e-04
   9   -240.5759773      2.591e-07      5.434e-05      4.030e-04
  10   -240.5762078      9.581e-07      2.391e-05      5.248e-04
  11   -240.5759770      9.593e-07      1.686e-05      4.910e-04
  12   -240.5760313      2.257e-07      8.626e-06      4.579e-04
  13   -240.5759432      3.663e-07      8.147e-06      3.143e-04
  14   -240.5760022      2.453e-07      5.269e-06      2.410e-04
  15   -240.5760397      1.558e-07      4.717e-06      9.087e-05
-----------------------------------------------------------------
SCF cycle converged

Final energies (Ha)

---------------------------------------------
Kinetic energy                 :   245.4007
    orbitals                   :   245.4007
    unbound ideal approx.      :     0.0000
Electron-nuclear energy        :  -587.2873
Hartree energy                 :   119.3529
Exchange-correlation energy    :   -18.0414
    exchange                   :   -17.0347
    correlation                :    -1.0067
---------------------------------------------
Total energy                   :  -240.5751
---------------------------------------------
Entropy                        :     2.7034
    orbitals                   :     2.7034
    unbound ideal approx.      :     0.0000
---------------------------------------------
Total free energy              :  -240.5777
---------------------------------------------

Chemical potential             :   0.699
Mean ionization state          :   3.000

Orbital eigenvalues (Ha) :

     |   n=l+1 |       2 |       3 |       4 |       5
-----+---------+---------+---------+---------+---------
 l=0 | -54.552 |  -3.446 |   0.356 |   2.925 |   6.879
   1 |  -2.070 |   0.700 |   3.194 |   6.954 |  11.912
   2 |   1.118 |   3.249 |   6.508 |  10.939 |  16.522


Orbital occupations (2l+1) * f\_\{nl\} :

     |   n=l+1 |       2 |       3 |       4 |       5
-----+---------+---------+---------+---------+---------
 l=0 |   2.000 |   2.000 |   2.000 |   0.000 |   0.000
   1 |   6.000 |   1.000 |   0.000 |   0.000 |   0.000
   2 |   0.000 |   0.000 |   0.000 |   0.000 |   0.000


func:'CalcEnergy' took: 18.9085 sec
    \end{Verbatim}

    In the above, at each step of the SCF (self-consistent field) cycle, the
spherically symmetric KS equations are solved for the chosen boundary
condition (Eq. (2) of the main paper). In atoMEC, we solve the KS
equations on a logarithmic grid to give more weight to the points
nearest the origin, i.e.~\(x=\log(r)\). Furthermore, we make a
transformation of the orbitals \(P_{nl}(x) = X_{nl}(x)\exp(x/2)\). Then
the equations to be solve become:

\begin{gather}
\frac{\textrm{d}^2 P_{nl}(x)}{\textrm{d}x^2} - 2e^{2x}(W(x)-\epsilon_{nl})P_{nl}(x)=0\,,\\
W(x) = v_\textrm{s}[n](x) + \frac{1}{2}\left(l+\frac{1}{2}\right)^2 e^{-2x}
\end{gather}

In atoMEC, we solve the KS equations using a matrix implementation of
Numerov's algorithm as described in
\href{https://aapt.scitation.org/doi/full/10.1119/1.4748813?casa_token=UMs6bxc3iB0AAAAA\%3AonvjnFq-KyEXZpEzUfGfyqQoNrMoP6AI0Wi7nrZrILOCM9Ah55XACGen5VLr-civFUtr2sVuCpw}{this
paper}. This means we diagonalize the following equation: \begin{align}
\hat{H}\vec{P} &= \vec{\epsilon} \hat{B} \vec{P} \\
\hat{H} &= \hat{T} + \hat{B} + W_\textrm{s}(\vec{x}) \\
\hat{T} &= -\frac{1}{2} e^{-2\vec{x}} \hat{A} \\
\hat{A} &= \frac{\hat{I}_{-1} -2\hat{I}_0 + \hat{I}_1}{\textrm{d}x^2} \\
\hat{B} &= \frac{\hat{I}_{-1} +10\hat{I}_0 + \hat{I}_1}{12}\,,
\end{align} where \(\hat{I}_{-1/0/1}\) are lower shift, identify and
upper shift matrices.

Since the Hamiltonian matrix \(H\) is sparse and we only seek the lowest
lying eigenvalues, there is no need to perform a full diagonalization
which scales with \(\mathcal{O}(N^3)\), with \(N\) being the size of the
radial grid. Instead, we use a sparse matrix diagonalization routine
from the \href{https://scipy.org/}{SciPy} library, which scales more
efficiently and allows us to go to larger grid sizes (and hence better
convergence).

After each step in the SCF cycle, the relative changes in the free
energy \(F\), density \(n\) and potential \(v_\textrm{s}\) are computed.
Specifically, the quantities computed are

\begin{align}
    \Delta F &= \left|\frac{F^{i}-F^{i-1}}{F^{i}}\right| \\
    \Delta n &= \frac{\int \mathrm{d}r|n^i(r)-n^{i-1}(r)|}{\int \mathrm{d}r n^i(r)}\\
    \Delta v &= \frac{\int \mathrm{d}r|v^i_\textrm{s}(r)-v_\textrm{s}^{i-1}(r)|}{\int \mathrm{d}r v_\textrm{s}^i(r)}
\end{align}

The values of these convergence parameters are controlled by the input
parameter \texttt{conv\_params} to the \texttt{CalcEnergy} function. For
example, if we wanted to reduce the required convergence, we could call:

\texttt{output=Al\_model.CalcEnergy(nmax,\ lmax,\ conv\_params=\{"econv":\ 1e-4,\ "nconv":\ 1e-3,\ "vconv":\ 1e-3\})}

At each stage of the SCF cycle, the KS potential is mixed with some
fraction of the KS potential from the previous iteration (which aids
convergence),
i.e.~\(v_\textrm{s}(r) = \alpha v^{i+1}_\textrm{s}(r) + (1-\alpha) v^i_\textrm{s}(r)\),
where \(\alpha\) is the mixing paramater. As seen above, this is
controlled via the \texttt{scf\_params} input parameter to the
\texttt{CalcEnergy} function. This dictionary also accepts a
\texttt{maxscf} key to terminate the SCF cycle after the requested
number of iterations.

At the end of the SCF cycle, various information is printed, such as the
breakdown of the total free energy and the KS eigenvalues and their
occupations. The printed `'Mean ionization state'' (MIS) output is
calculated using the threshold method described in the main paper. We
shall later see how to compute the MIS in atoMEC via the other methods
described in the paper. The output of the \texttt{CalcEnergy} function
is a dictionary containing information about the energy, density,
potential and orbtials. For example, we can extract and plot the density
from this output:

    \begin{tcolorbox}[breakable, size=fbox, boxrule=1pt, pad at break*=1mm,colback=cellbackground, colframe=cellborder]
\prompt{In}{incolor}{4}{\boxspacing}
\begin{Verbatim}[commandchars=\\\{\}]
\PY{c+c1}{\PYZsh{} import matplotlib for plotting and numpy for analysis}
\PY{k+kn}{import} \PY{n+nn}{matplotlib}\PY{n+nn}{.}\PY{n+nn}{pyplot} \PY{k}{as} \PY{n+nn}{plt}
\PY{k+kn}{import} \PY{n+nn}{numpy} \PY{k}{as} \PY{n+nn}{np}
\end{Verbatim}
\end{tcolorbox}

    \begin{tcolorbox}[breakable, size=fbox, boxrule=1pt, pad at break*=1mm,colback=cellbackground, colframe=cellborder]
\prompt{In}{incolor}{5}{\boxspacing}
\begin{Verbatim}[commandchars=\\\{\}]
\PY{c+c1}{\PYZsh{} extract the density object}
\PY{n}{dens\PYZus{}Al} \PY{o}{=} \PY{n}{output}\PY{p}{[}\PY{l+s+s2}{\PYZdq{}}\PY{l+s+s2}{density}\PY{l+s+s2}{\PYZdq{}}\PY{p}{]}

\PY{c+c1}{\PYZsh{} get the total density and grid}
\PY{n}{dens\PYZus{}tot} \PY{o}{=} \PY{n}{dens\PYZus{}Al}\PY{o}{.}\PY{n}{total}\PY{p}{[}\PY{l+m+mi}{0}\PY{p}{]}
\PY{n}{xgrid} \PY{o}{=} \PY{n}{dens\PYZus{}Al}\PY{o}{.}\PY{n}{\PYZus{}xgrid}  \PY{c+c1}{\PYZsh{} log grid}
\PY{n}{rgrid} \PY{o}{=} \PY{n}{np}\PY{o}{.}\PY{n}{exp}\PY{p}{(}\PY{n}{xgrid}\PY{p}{)}  \PY{c+c1}{\PYZsh{} radial grid}

\PY{c+c1}{\PYZsh{} plot the total density}
\PY{n}{plt}\PY{o}{.}\PY{n}{plot}\PY{p}{(}\PY{n}{rgrid}\PY{p}{,} \PY{n}{rgrid} \PY{o}{*}\PY{o}{*} \PY{l+m+mi}{2} \PY{o}{*} \PY{n}{dens\PYZus{}tot}\PY{p}{)}

\PY{c+c1}{\PYZsh{} some formatting}
\PY{n}{plt}\PY{o}{.}\PY{n}{xlim}\PY{p}{(}\PY{l+m+mi}{0}\PY{p}{,}\PY{l+m+mi}{3}\PY{p}{)}
\PY{n}{plt}\PY{o}{.}\PY{n}{ylim}\PY{p}{(}\PY{l+m+mi}{0}\PY{p}{,}\PY{l+m+mf}{1.2}\PY{p}{)}
\PY{n}{plt}\PY{o}{.}\PY{n}{xlabel}\PY{p}{(}\PY{l+s+sa}{r}\PY{l+s+s1}{\PYZsq{}}\PY{l+s+s1}{\PYZdl{}a\PYZus{}0\PYZdl{}}\PY{l+s+s1}{\PYZsq{}}\PY{p}{)}
\PY{n}{plt}\PY{o}{.}\PY{n}{ylabel}\PY{p}{(}\PY{l+s+sa}{r}\PY{l+s+s1}{\PYZsq{}}\PY{l+s+s1}{\PYZdl{}r\PYZca{}2 n(r)}\PY{l+s+s1}{\PYZbs{}}\PY{l+s+s1}{ (a\PYZus{}0)\PYZca{}}\PY{l+s+s1}{\PYZob{}}\PY{l+s+s1}{\PYZhy{}1\PYZcb{}\PYZdl{}}\PY{l+s+s1}{\PYZsq{}}\PY{p}{)}
\PY{n}{plt}\PY{o}{.}\PY{n}{show}\PY{p}{(}\PY{p}{)}
\end{Verbatim}
\end{tcolorbox}

    \begin{center}
    \adjustimage{max size={0.9\linewidth}{0.9\paperheight}}{Supplementary information_files/Supplementary information_8_0.png}
    \end{center}
    { \hspace*{\fill} \\}
    
    \hypertarget{convergence-testing}{%
\subsection{Convergence testing}\label{convergence-testing}}

There are various parameters which should be checked for convergence.
For all the boundary conditions, the main ones are:

\begin{itemize}
\tightlist
\item
  \texttt{nmax}: the maximum number of the principal quantum number
  \texttt{n}
\item
  \texttt{lmax}: the maximum number of the angular quantum number
  \texttt{l}
\item
  \texttt{grid\_params}: dictionary parameter controlling the
  logarithmic grid, in particular the number of grid points
  \texttt{ngrid}
\end{itemize}

Furthermore, for the \texttt{bands} boundary condition, there is the
additional dictionary parameter \texttt{band\_params}. The important
property within this is \texttt{nkpts} number of `k' points, which is
the number of states that are computed within all energy bands (spaced
linearly in energy) in our model.

The convergence for the \texttt{nmax} and \texttt{lmax} paramaters can
generally be chosen by eye, in other words by ensuring there are
sufficient states such that the highest levels have (nearly) zero
occoupations. Let us consider our Aluminium atom from earlier, but now
we shall increase the temperature (so more orbitals are required).

    \begin{tcolorbox}[breakable, size=fbox, boxrule=1pt, pad at break*=1mm,colback=cellbackground, colframe=cellborder]
\prompt{In}{incolor}{10}{\boxspacing}
\begin{Verbatim}[commandchars=\\\{\}]
\PY{c+c1}{\PYZsh{} set the temperature to 10 eV}
\PY{n}{Al\PYZus{}atom}\PY{o}{.}\PY{n}{units\PYZus{}temp}\PY{o}{=}\PY{l+s+s2}{\PYZdq{}}\PY{l+s+s2}{eV}\PY{l+s+s2}{\PYZdq{}}
\PY{n}{Al\PYZus{}atom}\PY{o}{.}\PY{n}{temp} \PY{o}{=} \PY{l+m+mi}{10}

\PY{c+c1}{\PYZsh{} run the calculation again}
\PY{n}{output} \PY{o}{=} \PY{n}{Al\PYZus{}model}\PY{o}{.}\PY{n}{CalcEnergy}\PY{p}{(}\PY{n}{nmax}\PY{p}{,} \PY{n}{lmax}\PY{p}{,} \PY{n}{write\PYZus{}info}\PY{o}{=}\PY{k+kc}{True}\PY{p}{)}
\end{Verbatim}
\end{tcolorbox}

    \begin{Verbatim}[commandchars=\\\{\}]
Starting SCF energy calculation

iscf   E\_free (Ha)    dE (1.0e-05)   dn (1.0e-04)   dv (1.0e-04)
-----------------------------------------------------------------
   0   -214.2614028      1.000e+00      9.999e-01      1.000e+00
   1   -223.8610709      4.288e-02      1.026e+00      7.418e-01
   2   -237.0092432      5.548e-02      8.081e-01      4.364e-01
   3   -242.4583716      2.247e-02      2.045e-01      7.027e-02
   4   -242.5740042      4.767e-04      3.631e-02      1.578e-02
   5   -242.5755381      6.323e-06      3.614e-03      6.087e-03
   6   -242.5765216      4.054e-06      1.355e-03      2.590e-03
   7   -242.5765568      1.453e-07      4.774e-04      1.238e-03
   8   -242.5765352      8.891e-08      1.697e-04      4.064e-04
   9   -242.5765542      7.835e-08      6.060e-05      3.080e-04
  10   -242.5765339      8.388e-08      2.207e-05      4.939e-05
-----------------------------------------------------------------
SCF cycle converged

Final energies (Ha)

---------------------------------------------
Kinetic energy                 :   245.1077
    orbitals                   :   245.1077
    unbound ideal approx.      :     0.0000
Electron-nuclear energy        :  -585.7123
Hartree energy                 :   118.9168
Exchange-correlation energy    :   -17.9809
    exchange                   :   -16.9756
    correlation                :    -1.0053
---------------------------------------------
Total energy                   :  -239.6687
---------------------------------------------
Entropy                        :     7.9127
    orbitals                   :     7.9127
    unbound ideal approx.      :     0.0000
---------------------------------------------
Total free energy              :  -242.5766
---------------------------------------------

Chemical potential             :   0.219
Mean ionization state          :   3.011

Orbital eigenvalues (Ha) :

     |   n=l+1 |       2 |       3 |       4 |       5
-----+---------+---------+---------+---------+---------
 l=0 | -54.623 |  -3.476 |   0.349 |   2.913 |   6.867
   1 |  -2.102 |   0.693 |   3.182 |   6.941 |  11.898
   2 |   1.112 |   3.237 |   6.496 |  10.927 |  16.509
   3 |   2.168 |   5.108 |   9.102 |  14.206 |  20.430
   4 |   3.254 |   6.876 |  11.539 |  17.285 |  24.134
   5 |   4.447 |   8.711 |  14.004 |  20.369 |  27.825
   6 |   5.759 |  10.653 |  16.561 |  23.529 |  31.580
   7 |   7.195 |  12.719 |  19.232 |  26.796 |  35.436


Orbital occupations (2l+1) * f\_\{nl\} :

     |   n=l+1 |       2 |       3 |       4 |       5
-----+---------+---------+---------+---------+---------
 l=0 |   2.000 |   2.000 |   0.825 |   0.001 |   0.000
   1 |   5.989 |   1.297 |   0.002 |   0.000 |   0.000
   2 |   0.809 |   0.003 |   0.000 |   0.000 |   0.000
   3 |   0.069 |   0.000 |   0.000 |   0.000 |   0.000
   4 |   0.005 |   0.000 |   0.000 |   0.000 |   0.000
   5 |   0.000 |   0.000 |   0.000 |   0.000 |   0.000
   6 |   0.000 |   0.000 |   0.000 |   0.000 |   0.000
   7 |   0.000 |   0.000 |   0.000 |   0.000 |   0.000


func:'CalcEnergy' took: 12.3132 sec
    \end{Verbatim}

    In the above example, we see that \texttt{nmax=5} seems to be
sufficiently large, but \texttt{lmax=2} is not, because the occupation
of the \(l=2,n=0\) state is 0.834, i.e.~significantly above zero. We
therefore choose a much larger value of \texttt{lmax} to check what is
required for convergence.

Since the diagonalization must be performed separately for every value
of \(0<l<\textrm{lmax}\), the computational time is proportional to
\texttt{lmax}. There is a simple kind of parallelization implemented in
atoMEC, courtesy of the
\href{https://joblib.readthedocs.io/en/latest/\#}{joblib} library, which
parallelizes the calculation over \texttt{l} (and also spin if
\texttt{model.spinpol=True}) and thus makes calculations more efficient.
This is enabled by the \texttt{config.numcores} parameter. Setting
\texttt{config.numcores=n} uses \texttt{n} cores; however, it is often
easiest to set \texttt{config.numcores=-1} which uses all the available
cores.

    \begin{tcolorbox}[breakable, size=fbox, boxrule=1pt, pad at break*=1mm,colback=cellbackground, colframe=cellborder]
\prompt{In}{incolor}{8}{\boxspacing}
\begin{Verbatim}[commandchars=\\\{\}]
\PY{c+c1}{\PYZsh{}enable parallelization}
\PY{k+kn}{from} \PY{n+nn}{atoMEC} \PY{k+kn}{import} \PY{n}{config}
\PY{n}{config}\PY{o}{.}\PY{n}{numcores} \PY{o}{=} \PY{o}{\PYZhy{}}\PY{l+m+mi}{1}

\PY{c+c1}{\PYZsh{} re\PYZhy{}run the calculation with larger lmax}
\PY{n}{lmax} \PY{o}{=} \PY{l+m+mi}{10}
\PY{n}{output} \PY{o}{=} \PY{n}{Al\PYZus{}model}\PY{o}{.}\PY{n}{CalcEnergy}\PY{p}{(}\PY{n}{nmax}\PY{p}{,} \PY{n}{lmax}\PY{p}{)}
\end{Verbatim}
\end{tcolorbox}

    \begin{Verbatim}[commandchars=\\\{\}]
Starting SCF energy calculation

iscf   E\_free (Ha)    dE (1.0e-05)   dn (1.0e-04)   dv (1.0e-04)
-----------------------------------------------------------------
   0   -214.2614028      1.000e+00      9.999e-01      1.000e+00
   1   -223.8610995      4.288e-02      1.026e+00      7.417e-01
   2   -237.0092428      5.548e-02      8.081e-01      4.361e-01
   3   -242.4583677      2.247e-02      2.045e-01      6.982e-02
   4   -242.5740042      4.767e-04      3.631e-02      1.574e-02
   5   -242.5755340      6.306e-06      3.614e-03      6.111e-03
   6   -242.5765476      4.179e-06      1.354e-03      2.676e-03
   7   -242.5765568      3.793e-08      4.785e-04      1.140e-03
   8   -242.5763752      7.487e-07      1.704e-04      1.049e-03
   9   -242.5765769      8.316e-07      6.034e-05      4.942e-04
  10   -242.5765766      1.312e-09      2.171e-05      2.100e-04
  11   -242.5765538      9.400e-08      8.506e-06      5.996e-04
  12   -242.5765538      4.726e-11      2.831e-06      2.414e-04
  13   -242.5765538      7.755e-12      1.036e-06      9.715e-05
-----------------------------------------------------------------
SCF cycle converged

Final energies (Ha)

---------------------------------------------
Kinetic energy                 :   245.1078
    orbitals                   :   245.1078
    unbound ideal approx.      :     0.0000
Electron-nuclear energy        :  -585.7124
Hartree energy                 :   118.9167
Exchange-correlation energy    :   -17.9809
    exchange                   :   -16.9756
    correlation                :    -1.0053
---------------------------------------------
Total energy                   :  -239.6687
---------------------------------------------
Entropy                        :     7.9127
    orbitals                   :     7.9127
    unbound ideal approx.      :     0.0000
---------------------------------------------
Total free energy              :  -242.5765
---------------------------------------------

Chemical potential             :   0.219
Mean ionization state          :   3.011

Orbital eigenvalues (Ha) :

     |   n=l+1 |       2 |       3 |       4 |       5
-----+---------+---------+---------+---------+---------
 l=0 | -54.624 |  -3.476 |   0.349 |   2.913 |   6.867
   1 |  -2.102 |   0.693 |   3.182 |   6.941 |  11.898
   2 |   1.112 |   3.237 |   6.496 |  10.927 |  16.508
   3 |   2.168 |   5.108 |   9.102 |  14.205 |  20.430
   4 |   3.254 |   6.876 |  11.538 |  17.285 |  24.134
   5 |   4.446 |   8.710 |  14.004 |  20.368 |  27.825
   6 |   5.758 |  10.653 |  16.561 |  23.529 |  31.580
   7 |   7.195 |  12.718 |  19.232 |  26.796 |  35.436


Orbital occupations (2l+1) * f\_\{nl\} :

     |   n=l+1 |       2 |       3 |       4 |       5
-----+---------+---------+---------+---------+---------
 l=0 |   2.000 |   2.000 |   0.825 |   0.001 |   0.000
   1 |   5.989 |   1.297 |   0.002 |   0.000 |   0.000
   2 |   0.809 |   0.003 |   0.000 |   0.000 |   0.000
   3 |   0.069 |   0.000 |   0.000 |   0.000 |   0.000
   4 |   0.005 |   0.000 |   0.000 |   0.000 |   0.000
   5 |   0.000 |   0.000 |   0.000 |   0.000 |   0.000
   6 |   0.000 |   0.000 |   0.000 |   0.000 |   0.000
   7 |   0.000 |   0.000 |   0.000 |   0.000 |   0.000


func:'CalcEnergy' took: 15.5121 sec
    \end{Verbatim}

    \begin{tcolorbox}[breakable, size=fbox, boxrule=1pt, pad at break*=1mm,colback=cellbackground, colframe=cellborder]
\prompt{In}{incolor}{ }{\boxspacing}
\begin{Verbatim}[commandchars=\\\{\}]

\end{Verbatim}
\end{tcolorbox}


    % Add a bibliography block to the postdoc
    
    
    
\end{document}
